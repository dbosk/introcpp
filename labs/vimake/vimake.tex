% $Id$
% Author: Daniel Bosk <daniel.bosk@miun.se>
\documentclass[a4paper]{miunasgn}
\usepackage[utf8]{inputenc}
\usepackage[T1]{fontenc}
\usepackage[english,swedish]{babel}
\usepackage{url,hyperref}
\usepackage{prettyref,varioref}
\usepackage{natbib}
\usepackage{color,listings}
\usepackage[nofancy,today]{svninfo}
\usepackage[natbib,varioref,prettyref,listings]{miunmisc}

\svnInfo $Id$
%\printanswers

\lstset{style=term}

\courseid{DT028G}
\course{Introduktion till programmering}
\assignmenttype{Laboration}
\title{Vim, GNU make och Hello World!}
\author{Daniel Bosk\footnote{%
	Detta verk är tillgängliggjort under licensen Creative Commons 
	Erkännande-DelaLika 2.5 Sverige (CC BY-SA 2.5 SE).
	För att se en sammanfattning och kopia av licenstexten besök URL 
	\url{http://creativecommons.org/licenses/by-sa/2.5/se/}.
}}
\date{\svnId}

\begin{document}
\maketitle
\thispagestyle{foot}
\tableofcontents


\section{Introduktion}
\noindent
För att kunna redigera din källkod krävs att du har tillgång till och kan 
använda en bra editor.
När du sedan är klar med kodskrivandet underlättar det att ha ett byggskript 
för att kompilera din kod till exekverbar kod.
Detta underlättar avsevärt när din kodbas växer, och det är därför bra att 
börja i tid.


\section{Syfte}
\noindent
Syftet med laborationen är att du ska bekanta dig med en bra editor och att du 
ska lära dig att skriva enkla byggskript för att hantera dina projekt.


\section{Läsanvisningar}
\label{sec:Prerequisites}
\noindent
För att genomföra laborationen behöver du orientera dig i språket C++, du bör 
alltså ha läst det första kapitlet i kursboken \cite{Deitel2012cht}.

Utöver detta ska du bekanta dig med GNU make(1).
Du börjar med detta genom att läsa de inledande delarna av dokumentationen 
\cite[avsnitt 1.1 till och med 2.2]{gnumake}.

Texteditorn vim(1) lär du dig genom att köra vimtutor(1), mer om detta under 
\prettyref{sec:Tasks}.

Som referens för de olika kommandona har du även manualsidorna för de olika 
kommandona; make(1), vim(1) och vimtutor(1).
Du öppnar dem genom kommandoraderna
\begin{terminal}
$ man 1 make
$ man 1 vim
\end{terminal}
som du skriver direkt i terminalen.


\section{Uppgift}
\label{sec:Tasks}
\noindent
Denna uppgift är uppdelad i två delar.
Den första handlar om att lära sig använda en riktig editor, nämligen vim(1).
Den andra delen handlar om att lära sig använda make(1) för att skapa 
byggskript som automatiskt kompilerar källkoden till objektkod och länkar dessa 
till en exekverbar fil.

\subsection{Vim}
\noindent
Det första du ska göra är att genomföra hela vimtutor(1).
Det är en praktisk tutorial för att snabbt komma igång med vim(1).
Du startar den genom att köra följande kommandorad:
\begin{terminal}
$ vimtutor
$
\end{terminal}

\subsection{GNU make}
\noindent
Ett byggskript för GNU make(1) är en vanlig textfil, vanligtvis döpt till 
\emph{Makefile}\footnote{%
	Detta underlättar eftersom att make(1) automatiskt tittar efter en fil döpt 
	till \emph{Makefile}, notera att namnet ska inledas med versal.
}.
Filen innehåller vad man kallar för mål\footnote{%
	\emph{Eng.} target.
}.
Målet är en fil som make(1) kommer att försöka skapa.
Varje mål har vad man kallar för förutsättningar eller beroenden\footnote{%
	\emph{Eng.} prerequisites, depends.
} som behövs för att skapa målet.
Därefter har målet ett recept\footnote{%
	\emph{Eng.} recipe.
} för att skapa målet utifrån beroendena.
Receptet består av skalkod, det vill säga vanliga kommandon som kan köras 
i terminalen.
Strukturellt ser det ut som följer.
\begin{source}
target: depend1 depend2
	shell commands to produce target from depends
	shell commands ...
\end{source}
Notera att skalkommandona måste vara indenterade med \emph{ett} tabbtecken.

make(1) kommer att undersöka om målet finns, om det inte finns ska det skapas.
Om målet finns och är äldre än någon av filerna som det beror på kommer det att 
skapas på nytt.
Om målet däremot finns och är nyare än alla beroenden kommer make(1) att säga 
att målet är färskt och inte behöver uppdateras.
På detta sätt behöver man inte bygga om ett helt projekt bara för att en 
källfil ändrats, det räcker då att skapa ny objektkod för den filen och sedan 
länka alla objektfiler igen.

För att skapa den exekverbara filen \texttt{program} från källfilen 
\texttt{main.cpp} kan vi ha följande kod i filen \texttt{Makefile}:
\begin{source}
program: main.cpp
	g++ -o program main.cpp
\end{source}
Därefter för att bygga vårt program och testköra skriver vi och får tillbaka 
följande rader i terminalen:
\begin{terminal}
$ make
g++ -o program main.cpp
$ ./program
Hello World!
$ make
make: `program' is up to date.
$
\end{terminal}
Notera att när vi försöker bygga programmet igen säger make(1) att 
\texttt{program} inte behöver byggas igen.

Att make(1) först tittar om målet finns som fil kan utnyttjas till andra saker.
Exemeplvis om man har ett mål vars recept aldrig skapar filen med målets namn 
kommer receptet att köras varje gång.
Detta gör att man kan skapa mål för att genomföra testning.
För att få make(1) att köra ett givet mål anges målets namn som argument till 
make(1), exempelvis:
\begin{terminal}
$ make test
[kör receptet för test]
$
\end{terminal}
Om inget mål anges som argument används det översta målet i \texttt{Makefile}.

Det du ska göra är att skriva ett ''Hello World!''-program och skriva en 
tillhörande \texttt{Makefile} med ett mål för programmet och ett mål med namnet 
\emph{test} för att testköra det.


\section{Examination}
\label{sec:Examination}
\noindent
Din lösning ska redovisas muntligen för en lärare vid något av kursens 
redovisningstillfällen.
När du redovisat och fått godkänt laddar du upp din källkod (och byggskript) 
till inlämningslådan i lärplattformen.



\bibliography{literature}
\end{document}
