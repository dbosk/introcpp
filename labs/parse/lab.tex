\documentclass[a4paper]{miunasgn}
\usepackage[utf8]{inputenc}
\usepackage[T1]{fontenc}
\usepackage[english,swedish]{babel}
\usepackage{url,hyperref}
\usepackage{prettyref,varioref}
\usepackage{natbib}
\usepackage{listings}
\usepackage[today,nofancy]{svninfo}
\usepackage{algorithm}
\usepackage{algpseudocode}
\usepackage[natbib,varioref,prettyref,listings,algorithm]{miunmisc}
%\renewcommand{\algorithmicrequire}{\textbf{Input:}}
%\renewcommand{\algorithmicensure}{\textbf{Output:}}
%Sample lab assignment, based on a Lab by Daniel Bosk. 

%\svnInfo $Id$
%\printanswers

\courseid{dt028g}
\course{Introduktion till programmering i C++}
\assignmenttype{Laboration}
\title{Fil IO, texthantering}
\author{Martin Kjellqvist\footnote{%
	E-post: \href{mailto:martin.kjellqvist@miun.se}{martin.kjellqvist@miun.se}.
}}
%\date{\svnId}

\begin{document}
\maketitle
\thispagestyle{foot}
%\tableofcontents

\section{Introduktion}
\label{sec:Introduktion}
\noindent
Det är ofta nödvändigt att formatera om data i filer.
Ibland görs det för läsbarhet, ibland görs det för att passa som indata till 
olika program.

\section{Syfte}
\label{sec:Syfte}
\noindent
Syftet med uppgiften är att ge färdighet i enkel texthantering och 
filhentering.
Uppgiften ligger till grund för filhenteringen i vidare labbuppgifter.

\section{Läsanvisningar}
\label{sec:Lasanvisningar}
\noindent
%\input{literature.tex}
Du ska vara klar med samtliga moment fram till och med filhantering. 

\section{Genomförande}
\label{sec:Genomforande}
\noindent
Du ska skapa ett fristående program som utför uppgifterna nedan.
Börja med att lösa uppgift ett och färdigställ den uppgiften innan du
börjar på uppgift 2.

Ditt program ska läsa filerna som omnämns ifrån ''current working directory''.
Samtliga filer är i textformat med ''\textbackslash{}n'' som radslut.
Du kan om du önskar använda pipes och redirects för att lösa filinläsningen och 
utskriften istället för \texttt{std::ifstream} och \texttt{std::ofstream}.
Kommentera i sådant fall det i koden tillsammans med exempel på hur programmet 
används.

\begin{questions}
	\question
	Filen \texttt{names.txt} innehåller namn och personnummer på följande form.
	Filen innehåller flera personer.
	\begin{verbatim}
Förnamn Efternamn
YYMMDDNNNN
Adressrad
	\end{verbatim}

	Din uppgift är att omvandla indatat till formen:
	\begin{verbatim}
Efternamn, Förnamn
Adressrad
	\end{verbatim}

	Filen \texttt{names.txt} finner du på:
	\begin{center}
		\url{http://w3.miun.se/dt028g/attach/164/names.txt}.
	\end{center}

	\question
	Utdatat ska indikera om personen är en man eller kvinna.
	Exempelvis för en man kan det stå:
	\begin{verbatim}
Efternamn, Förnamn [M]
Adressrad
	\end{verbatim}
	
\end{questions}


%%% EXAMINATION %%%
\section{Examination}
\label{sec:Examination}
\noindent
Din lösning ska redovisas muntligen för en lärare vid något av kursens 
redovisningstillfällen.
När du redovisat och fått godkänt laddar du upp din källkod (och byggskript) 
till inlämningslådan i lärplattformen.



\section{Algoritmer}
\label{sec:Algoritmer}
Googla algoritmen för att avgöra om en person är man eller kvinna.
En ledning är att du inte kan se det på namnet.
En del av problemet är att lista sig till hur man avgör om en person är man 
eller kvinna.


%\bibliography{literature}
\end{document}

