\documentclass[a4paper]{miunasgn}
\usepackage[utf8]{inputenc}
\usepackage[T1]{fontenc}
\usepackage[english,swedish]{babel}
\usepackage{url,hyperref}
\usepackage{prettyref,varioref}
\usepackage{natbib}
\usepackage{listings}
\usepackage[today,nofancy]{svninfo}
\usepackage{algorithm}
\usepackage{algpseudocode}
\usepackage[natbib,varioref,prettyref,listings,algorithm]{miunmisc}
%\renewcommand{\algorithmicrequire}{\textbf{Input:}}
%\renewcommand{\algorithmicensure}{\textbf{Output:}}
%Sample lab assignment, based on a Lab by Daniel Bosk. 

%\svnInfo $Id$
%\printanswers

\courseid{dt028g}
\course{Introduktion till programmering i C++}
\assignmenttype{Laboration}
\title{Sortering}
\author{Martin Kjellqvist\footnote{%
	E-post: \href{mailto:martin.kjellqvist@miun.se}{martin.kjellqvist@miun.se}.
}}
%\date{\svnId}

\begin{document}
\maketitle
\thispagestyle{foot}
%\tableofcontents

\section{Introduktion}
\label{sec:Introduktion}
\noindent
Sortering är en väldigt vanlig uppgift för datorprogram.
Metoderna för sortering är många och av varierande kvalitet.
Ofta introducerar kurser metoder som bubblesort och selection sort, de är 
förvisso enkla att förstå men är i övrigt av usel kvalitet. 

En metod som däremot är väldigt användbar är merge sort.
Den består i att upprepa en operation som kallas merge.
Den operationen undersöks i labben.

\section{Syfte}
\label{sec:Syfte}
\noindent
Du kommer i labben att använda dig av iteration och selektion med
filhantering.

\section{Läsanvisningar}
\label{sec:Lasanvisningar}
\noindent
Du ska ha färdigställt den tidigare labben innan du fortsätter med denna.  

\section{Genomförande}
\label{sec:Genomforande}
\noindent
Du ska för varje uppgift skapa en headerfil med implementation som utför
uppgiften.
Tillsammans med header och implementation skapar du ett testprogram
vars main-funktion demonstrerar att dina funktioner fungerar som de ska.

\begin{questions}
	\question
	En fil innehåller heltal.
	Du skall författa en funktion som avgör om heltalen i filen är i ordning 
	eller inte.
	Testprogrammet ska avgöra om filen A1 nedan är sorterad eller inte.
  En algoritmisk beskrivning ges i \prettyref{alg:isSorted}.
	Försök att komma på en metod på egen hand innan du tittar på algoritmen.

  \begin{algorithm}
    \caption{Avgör om en fil är sorterad}\label{alg:isSorted}
    \begin{algorithmic}
      \Require Filen A innehåller numeriska värden
      \Ensure Filen A är sorterad
      \State{ $a\gets readValue(A)$ }
      \While{ $notEndOfFile(A)$ }
        \State{ $b\gets readValue(A)$ }
        \If{ $a > b$}
          \State \Return false
        \EndIf
        \State $a\gets b$
      \EndWhile
      \State \Return true
    \end{algorithmic}
  \end{algorithm}

	\question 
	Två filer, A och B är sorterade.
	(Kontrollera detta med den funktion du författat i uppgiften ovan.)
	Utifrån filerna ska du skapa en fil som innehåller samtliga element i
	sorterad ordning.
	Denna operation kallas för en merge.
  En algoritmisk beskrivning ges i \prettyref{alg:merge}.
	Försök att implementera den på egen hand innan du tittar på algoritmen.

  \begin{algorithm}
    \caption{Merge med två filer}\label{alg:merge}
    \begin{algorithmic}
      \Require Filerna A och B är sorterade enligt algoritm~\ref{alg:isSorted}
      \Ensure Filen C innehåller samtliga värden från A och B i sorterad ordning
      \State{ $a\gets readValue(A)$ }
      \State{ $b\gets readValue(B)$ }
      \While{ $notEndOfFile(A)$ And $notEndOfFile(B)$}
        \Comment{ Avgör vilket värde som ska skrivas till C}
        \If{ $a < b$ }
          \State{Skriv a till C}
          \State{ $a\gets readValue(A)$ }
        \Else
          \State{Skriv b till C}
          \State{ $b\gets readValue(B)$ }
        \EndIf
      \EndWhile
      \Comment A eller B är slut, skriv klart båda
      \While{ $notEndOfFile(A)$ }
        \State{Skriv a till C}
        \State{ $a\gets readValue(A)$ }
      \EndWhile
      \While{ $notEndOfFile(B)$ }
        \State{Skriv b till C}
        \State{ $b\gets readValue(B)$ }
      \EndWhile
    \end{algorithmic}
  \end{algorithm}

\end{questions}

Filen A hittar då på
\begin{center}
	\url{http://w3.miun.se/dt028g/attach/161/A}.
\end{center}

Filen B finner du på
\begin{center}
	\url{http://w3.miun.se/dt028g/attach/162/B}.
\end{center}

Filen A1 finner du på
\begin{center}
	\url{http://w3.miun.se/dt028g/attach/163/A1}.
\end{center}


%%% EXAMINATION %%%
\section{Examination}
\label{sec:Examination}
\noindent
Din lösning ska redovisas muntligen för en lärare vid något av kursens 
redovisningstillfällen.
När du redovisat och fått godkänt laddar du upp din källkod (och byggskript) 
till inlämningslådan i lärplattformen.


%\bibliography{literature}
\end{document}
