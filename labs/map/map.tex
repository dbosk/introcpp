% $Id$
% Author: Daniel Bosk <daniel.bosk@miun.se>
\documentclass[a4paper]{miunasgn}
\usepackage[utf8]{inputenc}
\usepackage[T1]{fontenc}
\usepackage[english,swedish]{babel}
\usepackage{url,hyperref}
\usepackage{prettyref,varioref}
\usepackage{natbib}
\usepackage{color,listings}
\usepackage[nofancy,today]{svninfo}
\usepackage[natbib,varioref,prettyref,listings]{miunmisc}

\svnInfo $Id$
%\printanswers

\courseid{DT028G}
\course{Introduktion till programmering}
\assignmenttype{Laboration}
\title{Att räkna antalet förekomster av ord i en text}
\author{Daniel Bosk\footnote{%
	Detta verk är tillgängliggjort under licensen Creative Commons 
	Erkännande-DelaLika 2.5 Sverige (CC BY-SA 2.5 SE).
	För att se en sammanfattning och kopia av licenstexten besök URL 
	\url{http://creativecommons.org/licenses/by-sa/2.5/se/}.
}}
\date{\svnId}

\begin{document}
\maketitle
\thispagestyle{foot}
\tableofcontents


\section{Introduktion}
\noindent
År 1986 gav Jon Bentley\footnote{%
	Skapare av bland annat datastrukturen \(k\)-dimensionella träd.
} en utmaning till Donald Knuth\footnote{%
	Bland annat fader till modern datalogi och skapare av \TeX.
}.
Utmaningen var att med \emph{literate programming} skapa ett program som skulle 
läsa innehållet i en text och bestämma de \(n\) mest förekommande orden och 
skriva ut en sorterad lista tillsammans med antalet förekomster av respektive 
ord.
Knuth löste naturligtvis problemet.
Likaså gjorde Douglas McIlroy\footnote{%
	Skapare av bland annat UNIX:s \emph{pipelines}, på svenska närmast kallat 
	rör.
}, fast med ett mycket kort skalskript.
McIlroys lösning följer:
\begin{src}
n=10
cat hitchhikersguide.txt | \
	tr -cs A-Za-z '\n' | \
	tr A-Z a-z | \
	sort | \
	uniq -c | \
	sort -rn | \
	head -n $n
\end{src}
Denna kan köras direkt i terminalen på ett UNIX-likt system och varje kommando 
har en manualsida där du kan läsa om vad kommandot gör med de givna argumenten.


\section{Syfte}
\noindent
Syftet med laborationen är att du ska få bekanta dig med ytterligare några 
C++-konstruktioner och tillämpa dessa för att lösa ett problem.
En av de C++-konstruktioner som är nya för denna laboration är 
\texttt{std::map}.


\section{Läsanvisningar}
\label{sec:Prerequisites}
\noindent
Du bör ha läst om följande datastrukturer i standardbiblioteket för C++:
\begin{itemize}
	\item \texttt{std::string},
	\item \texttt{std::vector}, och
	\item \texttt{std::map}.
\end{itemize}
Därefter bör du ha läst om följande funktioner från standardbibliotekets 
algoritmdel:
\begin{itemize}
	\item \texttt{std::transform},
	\item \texttt{std::sort},
	\item \texttt{std::find},
	\item \texttt{std::count},
	\item \texttt{std::max\_element},
	\item \texttt{std::unique}.
\end{itemize}
Följande funktioner från C-biblioteket kan också vara behjälpliga:
\begin{itemize}
	\item tolower(3),
	\item toupper(3).
\end{itemize}

Du kan läsa om standardbiblioteket i kurslitteraturen \cite{Deitel2012cht} 
eller någon referens, exempelvis \emph{C++ Reference} \cite{cppref}, eller 
manualsidorna.


\section{Uppgift}
\label{sec:Tasks}
\noindent
Du ska lösa Bentleys problem till Knuth.
Som indata har du en textfil innehållandes ett kortare utdrag ur \emph{The 
Hitchhiker's Guide To The Galaxy}, du finner den på lärplattformen.
Utdata från ditt program ska vara de \(n\) mest förekommande orden sorterade 
i ordningen mest förekommande först.
Det variabla antalet \(n\) ska enkelt kunna ändras i programmet.


\section{Examination}
\label{sec:Examination}
\noindent
Din lösning ska redovisas muntligen för en lärare vid något av kursens 
redovisningstillfällen.
När du redovisat och fått godkänt laddar du upp din källkod (och byggskript) 
till inlämningslådan i lärplattformen.



\bibliography{literature}
\end{document}
