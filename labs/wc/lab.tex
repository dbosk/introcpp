\documentclass[a4paper]{miunasgn}
\usepackage[utf8]{inputenc}
\usepackage[T1]{fontenc}
\usepackage[english,swedish]{babel}
\usepackage{url,hyperref}
\usepackage{prettyref,varioref}
\usepackage{natbib}
\usepackage{listings}
\usepackage[today,nofancy]{svninfo}
\usepackage{algorithm}
\usepackage{algpseudocode}
\usepackage[natbib,varioref,prettyref,listings,algorithm]{miunmisc}
%Sample lab assignment, based on a Lab by Daniel Bosk. 

%\svnInfo $Id$
%\printanswers

\courseid{dt028g}
\course{Introduktion till programmering i C++}
\assignmenttype{Laboration}
\title{Iteration och aritmetik}
\author{Martin Kjellqvist\footnote{%
	E-post: \href{mailto:martin.kjellqvist@miun.se}{martin.kjellqvist@miun.se}.
}}
%\date{\svnId}

\begin{document}
\maketitle
\thispagestyle{foot}
%\tableofcontents

\section{Introduktion}
\label{sec:Introduktion}
\noindent
Det är enkelt att behandla stora datamängder med hjälp av iterationssatser.  
Labben illustrerar hur det kan gå till under gynnsamma omständigheter.

\section{Syfte}
\label{sec:Syfte}
\noindent
Syftet med uppgiften är att demonstrera iterationssatser och enkla beräkningar.  
\section{Läsanvisningar}
\label{sec:Lasanvisningar}
\noindent
%\input{literature.tex}
Du ska vara klar med samtliga moment fram till och med iterationer. 

\section{Genomförande}
\label{sec:Genomforande}
\noindent
Du ska skapa ett fristående program som utför följande.
Filen A innehåller heltal.
Ur värdena i A ska du ta fram medelvärde, max- och minimumvärde.

Filen A finner du på
\begin{center}
	\url{http://w3.miun.se/dt028g/attach/165/values.txt}.
\end{center}
	
\section{Examination}
\label{sec:Examination}
\noindent
Din lösning ska redovisas muntligen för en lärare vid något av kursens 
redovisningstillfällen.
När du redovisat och fått godkänt laddar du upp din källkod (och byggskript) 
till inlämningslådan i lärplattformen.

\end{document}
