% $Id$
\documentclass[a4paper,portrait]{miunart} % students are not allowed to use the
										% university logo
\usepackage[english,swedish]{babel}
\usepackage[utf8]{inputenc}
\usepackage[T1]{fontenc}
\usepackage{natbib}
% \usepackage{SIunits}
\usepackage{listings}
\usepackage{varioref,prettyref}
\usepackage{url,hyperref}
\usepackage[varioref,prettyref,natbib,listings]{miunmisc}

\title{Exempelsamling - idiomatisk c++}
\author{Martin Kjellqvist\footnote{%
	E-post: \href{mailto:martin.kjellqvist@miun.se}{martin.kjellqvist@miun.se}.
	Ämne: C++.
}}
\date{den 18 Augusti 2014}

\lstset{language=C++}
\begin{document}
\maketitle
\begin{abstract}
	\noindent
	Denna exempelsamling har för avsikt att vara en samling idiomatiska och 
	välskrivna kodsnuttar för att lösa vanliga problem. Problem som man kan 
	tänkas stöta på under någon av c++ kurserna som ges på universitetet.  
	\keywords{c++, c++11, problemlösning, algoritmer}
\end{abstract}


\section{Introduktion}
\label{sec:intro}
\noindent
Avsikten med denna exempelsamlig är flerfaldig. Huvudsyftet är att ha exempel 
att peka på i undervisningen på Mittuniversitetet. Som student kan man gärna 
använda dessa exempel som mallar för egen kod i projekt och labbar. Man måste 
då naturligtvis referera till denna exempelsamling och det exempel man använt. 

Texten är skriven för den som vill lära sig att programmera i c++ och c++ 
11. I framtiden kommer tillägg att göras för c++ 14. 

Jag beskriver samlingen som idiomatisk då jag har strävat efter att använda de 
c++ mekanismer som \emph{bör} användas, hellre än att göra exemplen så enkla 
som möjligt. I flera exempel existerar inte idiomatiska lösningar. 

Kommentarer till exemplen är avsedda att motivera stil och struktur samt att 
beskriva idiom.

Samlingen spänner över hela språket.  Samlingen är således aktuell under hela 
din utbildning.

Exempelsamlingen är avsedd att komplettera föreläsningsmaterial och 
labbinstruktioner.  Om ni någonsin får nöjet att bevittna någon av författarens 
föreläsningar på området kommer ni förmodligen märka likheter med något exempel 
här. 

Jag ger referenser till exempel som jag inte betraktar som triviala, eller som 
inte är direkt härledda ur c++ standarden.

Stilen som förespråkas är en utökning av K\&R-style.
\pagebreak
\tableofcontents
\pagebreak
\section{Grundläggande byggstenar}
\label{sec:basics}
\noindent
Exempel som kan användas för en c++ introduktion.
\subsection{Hello World}
\label{sec:helloworld}
\lstinputlisting[caption={Hello world}]{1.1-hello-world.cpp}

Signaturen för main måste enligt standarden vara \texttt{int main()} eller 
\texttt{int main(int argc, char* argv[])}.  Kompilatorer kan tillåta 
ytterligare vaianter på \texttt{main} men det är sällsynt (§3.6.1(2))
\cite{ISO:n3337}. 

Man bör fördera att uttryckligen specificera namnrymden \texttt{std::}. Det 
ökar läsbarheten och det är en god vana. Se även \ref{sec:class.headers}.

Operatorer ska omges av mellanslag för läsbarhet. Parenteser kramar innehållet.

\texttt{using namespace std;} som ses i många läroböcker är ett ofog i de allra 
flesta situationer. 

%\note{Hello hello world}
\pagebreak
\subsection{Headerfiler}
\label{sec:headers}
En homogen form för headerfiler är viktig.  Headerfilen kommer läsas av alla 
som vill använda dina funktioner i framtiden. Headerfilen används också för att 
generera dokumentation. 

\lstinputlisting[caption={Enkel header}]{1.2-headers.h}


\emph{Include guards} skrivs på formen \texttt{\#ifndef \_HEADERNAMN}. Använd 
inte \texttt{\#pragma once}-direktivet. Det är kompilatorberoende (även om 
många kompilatorer stödjer det).

Man ska under inga omständigheter föra in andra namn än de som definieras av 
headern i scopet av en headerfil. Det är en oönskad och oväntad sidoeffekt.  
Därför anges alla typer som inte definieras i headern med sitt fulla namn, 
exempelvis \texttt{std::string}. \emph{using}-deklarationer ska alltså undvikas 
då det introducerar namn i onödan.

Kommentera funktioner i header enligt det format som bjuds av 
dokumentationssystemet.

Om man har många headerfiler i sitt projekt är det lämpligt att man ger dem en 
hierarkisk struktur i filsystemet. 

En headerfil kompileras aldrig fristående, utan är alltid en del av en 
\emph{translation unit}. (§2.1) \cite{ISO:n3337}.

\pagebreak
\subsection{Headerfiler för klasser}
\label{sec:class.headers}
En header för en klassdefinition har en liknande struktur som en header som 
innehåller funktioner.  Man bör som regel undvika att blanda funktioner och 
klassdefinitioner. Undantag görs då funktionen är nära relaterad till 
klassdefinitionen (exempelvis operatoröverlagringar). 

\lstinputlisting[caption={Klassheader}]{1.3-class-headers.h}

\emph{Include guards} är helt nödvändiga för att uppfylla
\emph{One definition rule} per §3.2, \cite{ISO:n3337}


Kommentarer måste inte överdrivas. Vanliga accessormetoder behöver inte 
minutiöst beskrivas med kommentarer, de är självklara hur de fungerar. 

Konstruerare bör deklareras \emph{explicit} om ens design inte kräver implicit 
omvandling av argumenten (§12.3.1) \cite{ISO:n3337}. Implicita omvandlingar kan 
ge väldigt bekväm syntax, men bör tillämpas med varsamhet.

Återigen, man ska under inga omständigheter föra in andra namn än de som 
definieras av headern i scopet av en headerfil. Det är en oönskad och oväntad 
sidoeffekt.  Därför anges alla typer som inte definieras i headern med sitt 
fulla namn, exempelvis \texttt{std::string}. \emph{using}-deklarationer ska 
alltså undvikas.

\pagebreak
\subsection{Headerfiler för templates}
\label{sec:class.template.headers}

En template-header innehåller som regel även implementationen av alla delar 
eftersom mallen instansieras i den translationsenhet som använder den.  Klassen 
kan inte kompileras innan instansiering om inte explicita instansieringar eller 
specialiseringar angivits, se §14.1 \cite{ISO:n3337}.

\lstinputlisting[caption={Header med mallklass}]{1.4-template-headers.h}

Vertikalt tomrum och Kommentarer har tagits bort för att undvika sidbrytning 
i exemplet.

Namnen på mallparametrarna har en etablerad konvention från 
standardbiblioteket. Exempelvis används \texttt{T} för enkla värdetyper, 
\texttt{PRED} för predikat ofta definerade som funktionsobjekt eller 
lambdauttryck. En vanlig specialisering av \texttt{PRED} är \texttt{COMP} 
vilket är en komparator som tar två argument av typen \texttt{T} och motsvarar 
semantiskt operationen \texttt{arg1 < arg2}. \texttt{IT} representerar en 
generell iterator. Vanliga specialiseringar beskrivs i §24.2.1 (2) 
\cite{ISO:n3337} och implementeras gärna med \emph{type\_traits}

\texttt{using value\_type = T;} i klassdefinitionen är användbart eftersom det 
gör det möjligt att hänvisa till malltypen \texttt{T} från en instans. Skriv en 
\texttt{using}-deklaration för varje mallparameter. Ge typen ett namn som väl 
berkriver dess betydelse.  SFINAE-typer, här 
\texttt{enable\_if<is\_numeric<T>::value, T>::type>} har naturligtvis 
ingen relaterad \texttt{typedef}. 

Funktionsmallen kommer tack vare specialiseringen \texttt{const point<T>\&} 
bara specialiseras för giltiga \texttt{point} instanser.

Headers för malltyper är mer svårlästa än motsvarande icke-mall. Man kan 
underlätta läsbarheten genom att för längre funktioner inte definiera 
funktionerna \emph{inline}. Notera att en \emph{inline}-definierad funktion 
påverkar kompilatorns inte optimeringsprocess. 


\pagebreak
\section{Nytt i C++11}
\label{sec:new-stuff}
auto, decltype, constexpr. using.
move,
lambda,
chrono,
random,
thread.
user-defined literals.

\pagebreak
\section{Iteration}
\label{sec:iteration}

De vanliga iterationskonstruktionerna med \emph{for} och \emph{while} har tack 
vare ett utförligt standardbiliotek har i en uppsjö av situationer fått mer 
läsbara alternativ. Range-based \emph{for} är också att föredra över vanlig 
\emph{for}.

Några vanliga situationer illustreras nedan.  

\subsection{for-satsen}
\label{sec:iteration:for}

Använd \emph{for} för olika typer av uppräkningar. Typiska tillämpningar är 
i synnerhet:
\begin{itemize}
	\item{Uppräkning av ett numeriskt intervall. $[0..10[$, $[10..0]$ etc.}
	\item{Uppräkning av några eller alla element i en \emph{container}.}
\end{itemize}
			
Strävan då man tecknar sina \emph{for}-satser är förstås att avsikt och 
funktion ska vara så tydig som möjligt. Med det i åtanke betraktas följande 
exempel:

\lstinputlisting[caption={for-sats}]{2.1-for.cpp}

Använd dessa former då uppräkningen explicit kräver uppräkningsindex; ofta 
behövs inte detta index. Ej teckensatta typer och i synnerhet \texttt{size\_t} 
bör användas för uppräkningar i containers.

Se till exempel följande exempel att beräkna medelvärdet i en \emph{container}.

\lstinputlisting[caption={for-sats för 
container}]{2.2-for-container-1.cpp}

Index används för element-access, men är i övrigt onödigt. Lösningen är direkt 
men onödigt pratig. Man kan även göra den mer generell med hjälp av iteratorer. 

\lstinputlisting[caption={for-sats för container med 
iteratorer}]{2.2-for-container-2.cpp}

\emph{sample1} är lång och otymplig. Typnamnet för iteratorn är ointressant och 
tillför inget för förståelsen av koden.  \emph{sample2} är betydligt mer läsbar 
tack vare att typnamnet för iteratorn ersätts av malltypen \texttt{IT}.  
\emph{sample2} illustrerar också hur man enkelt kan omvandla en iteratorbaserad 
funktion till en containerbaserad funktion.

Avsikten att iterera över alla element i vektorn är uttryckt på liknande sätt 
i alla tre ovanstående exempel.

För att slippa det otympliga typnamnet kan låta kompilatorn deducera typen med 
hjälp av \emph{auto}. (§7.1.6.4) \cite{ISO:n3337}

\lstinputlisting[caption={for-sats för container med 
auto}]{2.2-for-container-auto.cpp}

Kompilatorn deducerar typen och vi kan använda iteratorn på vanligt sätt.

\emph{Range}-baserade \emph{for}-satser gör det ännu enklare att iterera över 
alla element. (§6.5) \cite{ISO:n3337}. En mallösning gör den än mer flexibel. 

\lstinputlisting[caption={range-baserad for-sats 
}]{2.2-for-container-range.cpp}

Alla vektorer vars \texttt{value\_type} stödjer \texttt{+=} kan summeras på 
detta sätt.  Man kan även tänka sig att ta alla iteratorbaserade containers med 
en helt generell malltyp och en SFINAE lösning baserad på existensen av 
medlemmen \texttt{T::iterator}. 

Valet mellan ovanstående alternativ ska baseras på läsbarhet. Samtliga 
alternativ har likvärdiga prestanda. 

Ofta finns det redan funktioner i standardbiblioteket för att handskas med 
vanliga operationer. Både \texttt{std::for\_each} och \texttt{std::accumulate} 
är värda att nämnas. 

\lstinputlisting[caption={for-satsalternativ 
i \texttt{std::}}]{2.2-for-stdlib.cpp}

\emph{sample1} uttrycker uppräkningen över samtliga element väldigt tydligt med 
\texttt{std::for\_each}. Lambdauttrycket fångar \texttt{sum} som referens och 
summerar som tidigare exempel. 

\emph{sample2} uttrycker summeringen i ett enkelt funktionsanrop. Notera att 
heltalsargumentet krävs för att \emph{std::accumulate} ska få en returtyp.  En 
returtyp kan inte deduceras.
\emph{sample3} illustrerar användningen av {std::accumulate} i de fall 
värdetypen är en malltyp. 

Samtliga exempel har likvärdig prestanda och i detta specifika fall är 
lösningen med \texttt{std::accumulate} att föredra. \texttt{<algorithm>} 
innehåller en uppsjö av olika mallar för att förenkla och tydliggöra avsikten 
av vanligt förekommande algoritmer.

\pagebreak[4]
Som avrundning demonstreras en lämplig implementation av en flexibel
medelvärdesberäkning.  
\lstinputlisting[caption={Medelvärdesberäkning}]{2.2-for-stdlib-final.cpp}

Tack vare funktionerna i \texttt{<algorithm>} och \texttt{<iterator>} är koden 
självförklarande och ger en medelvärdesberäkning som fungerar för alla 
upptänkliga konstruktioner, iteratorer, vanliga pekare och arrayer. Alla 
containers i standardbiblioteket kan också användas. 

\pagebreak
\section{Sortering}
\label{sec:sorting}
I detta avsnitt behandlas olika sätt att nyttja \texttt{std::sort}, samt en 
radix-implementation för heltalssortering. 
\subsection{std::sort}
\label{sec:std::sort}

Standarden kräver $O(NlogN)$ §25.4.1.1.1(3) \cite{ISO:n3337} för alla indata.
En vanlig och högpresterande implementation är introsort  \cite{intro-sort}.
\bibliographystyle{plain}
\bibliography{references}
\end{document}
