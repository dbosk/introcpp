% $Id$
% Author: Martin Kjellqvist <martin.kjellqvist@miun.se>
\documentclass[a4paper]{miunasgn}
\usepackage[utf8]{inputenc}
\usepackage[english]{babel}
\usepackage[hyphens]{url}
\usepackage{hyperref}
\usepackage{prettyref,varioref}
\usepackage{natbib}
\usepackage{color,listings}
\usepackage[nofancy,today]{svninfo}
\usepackage[natbib,varioref,prettyref,listings]{miunmisc}

\svnInfo $Id$
%\printanswers

\courseid{DT028G}
\course{Introduktion till programmering}
\assignmenttype{Projekt}
\title{algoritmer i std::}
\author{Martin Kjellqvist\footnote{%
  Detta verk är tillgängliggjort under licensen Creative Commons 
  Erkännande-DelaLika 2.5 Sverige (CC BY-SA 2.5 SE).
	För att se en sammanfattning och kopia av licenstexten besök URL 
	\url{http://creativecommons.org/licenses/by-sa/2.5/se/}.
}}
\date{\svnId}

\begin{document}
\maketitle
\thispagestyle{foot}
\tableofcontents


\section{Introduktion}
\noindent
Standardbiblioteket innehåller många intressanta algoritmer som vi inte haft
utrymme att fördjupa oss i.  


\section{Syfte}
\noindent
Detta projekt består av en orientering och fördjupning av de vanligaste algoritmerna i std::.

Du kommer även att undersöka iteratorer för filströmmar.

\section{Teori}
\label{sec:Theory}
\noindent
Dokumentation för samtliga strukturer och funktioner hittar ni på \href{www.cplusplus.com}.

\section{Uppgift}
\label{sec:Tasks}
\noindent
Projektet är uppdelat i olika betygssteg.
För betyget D krävs att du genomför grunduppgiften utan anmärkningar 
enligt granskningsprotokollet.
Vid maximalt två anmärkningar ges betyget E, vid fler anmärkningar betyget F.

För betygen C-A krävs de extrauppgifter som ges efter grunduppgiften, för ett 
betyg krävs att samtliga föregående extrauppgifter även är genomförda.
(Även dessa ska vara utan anmärkningar, vid anmärkning sänks betyget 
fortfarande till E.)


För varje deluppgift ska du förutom kod ge en beskrivning av 
de inblandade funktionerna. Beskrivningen ska förklara din kod. 
Ge beskrivningen som kommentarer i din kod.

\subsection{Grunduppgift}
\noindent
Skapa en \code{vector<int>} innehållande 1\,000\,000 slumpelement använd 
funktionerna \code{generate} och \code{rand}.
Sortera elementen med \code{sort}.

Undersök med funktionen \code{find} om något av elementen 0, 1 eller 20 finns 
i din vector.
Undersök även med funktionen \code{binary_search} om något av elementen 0, 
1 eller 20 finns i din vector.
Beskriv skillnaden mellan \code{find} och \code{binary_search}.

Funktionerna hittar du i \code{<algorithm>}.

\subsection{Extrauppgift C}
\noindent
Undersök om det finns dubletter i din vector. Använd \code{adjacent_find}.
Skriv ut det första dublettvärdet i din serie. Skriv ut det sista
dublettvärdet. 

Beräkna summan av alla element i din vector. Använd \code{accumulate}.

Negera alla värden i din vector. Använd \code{transform} och \code{negate}.
Negate finns i \code{<functional>}.

\subsection{Extrauppgift B}
\noindent
Demonstrera funktionen \code{copy} genom att kopiera samtliga värden
från din vector till en list. Använd \code{back_insert_iterator}.

Demonstrera \code{copy} genom att kopiera din vectors värden till en utfil $A$. 
Använd \code{ostream} och \code{ostream_iterator}. Se \code{<iterator>}. 

\subsection{Extrauppgift A}
\noindent
Skriv en mall-funktion som implementerar shellsort \cite{Wikipedia2012s}.
Funktionen ska ha följande deklaration:
\begin{src}
template <typename T>
void shellsort(T begin, T end);
\end{src}
Använd hoppen som beskrivs i \cite{clura2001shellsortavg}.
Dessa hopp finns även angivna i \cite{Wikipedia2012s}.

%%% EXAMINATION %%%
\section{Examination}
\label{sec:Examination}
\noindent
Din lösning ska redovisas muntligen för en lärare vid något av kursens 
redovisningstillfällen.
När du redovisat och fått godkänt laddar du upp din källkod (och byggskript) 
till inlämningslådan i lärplattformen.



\bibliography{literature}
\end{document}
