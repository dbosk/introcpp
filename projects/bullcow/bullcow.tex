% $Id$
% Author: Daniel Bosk <daniel.bosk@miun.se>
\documentclass[a4paper]{miunasgn}
\usepackage[utf8]{inputenc}
\usepackage[T1]{fontenc}
\usepackage[english,swedish]{babel}
\usepackage[hyphens]{url}
\usepackage{hyperref}
\usepackage{prettyref,varioref}
\usepackage{natbib}
\usepackage{color,listings}
\usepackage{algorithm}
\usepackage{algpseudocode}
\usepackage{amsmath,amsthm,amssymb}
\usepackage[nofancy,today]{svninfo}
\usepackage[natbib,varioref,prettyref,listings,algorithm]{miunmisc}

\svnInfo $Id$
%\printanswers

\theoremstyle{definition}
\newtheorem{example}{Exempel}

\courseid{DT028G}
\course{Introduktion till programmering}
\assignmenttype{Projekt}
\title{Ett härligt parti \emph{Bulls and Cows}}
\author{Daniel Bosk\footnote{%
	Detta verk är tillgängliggjort under licensen Creative Commons 
	Erkännande-DelaLika 2.5 Sverige (CC BY-SA 2.5 SE).
	För att se en sammanfattning och kopia av licenstexten besök URL 
	\url{http://creativecommons.org/licenses/by-sa/2.5/se/}.
}}
\date{\svnId}

\begin{document}
\maketitle
\thispagestyle{foot}
\tableofcontents


\section{Introduktion}
\noindent
\emph{Bulls and Cows}, av vilket \emph{Mastermind} är en känd variant, är ett 
klassiskt kodknäckningsspel för två spelare där det går ut på att gissa sig 
vilket tal som motståndaren tänker på.
De två rollerna i spelet är alltså en \emph{kodmakare}, som hittar på talet, 
och en \emph{kodknäckare}, som ska lista ut talet.

Spelet fungerar enligt följande.
Jag tänker på ett fyrsiffrigt tal där varje siffra endast får förekomma en 
gång, upprepningar är alltså inte tillåtna, och du ska gissa på vilket tal det 
är.
Utifrån din gissning svarar jag med ett antal \emph{bulls} eller \emph{cows}.
Du får en bull för varje siffra i talet som är på rätt plats och en cow för 
varje siffra som är rätt men på fel plats.
(Notera att du får inte veta vilka tal som gav dig bull eller cow.)
För siffror som inte är med i talet får du ingenting.
Därefter får du gissa igen utifrån den information du har fått.
Detta illustreras enklast med ett exempel.

\begin{example}\label{ex:BullCow}
	Jag tänker på talet 1234.
	Din första gissning är 1437.
  Du får då som svar två bulls för ettan och trean, och du får en
  cow för fyran.
  Du gissar härnäst på 2437 och får då svaret en bull för trean och två cows 
  för tvåan och fyran.
  Sedan fortsätter spelet tills du fått fyra bulls som svar.
\end{example}

Notera återigen att du inte får veta för vilka siffror du får bulls respektive 
cows, detta får du veta här enbart för att exemplet ska vara tydligt.

Det finns lite olika angripssätt för att lista ut talet, se 
\prettyref{sec:Theory}, och i detta projekt kommer du att titta på ett av dem, 
men fokus kommer att ligga på att implementera spelet.

Notera att Mastermind och Bulls and Cows skiljer sig något.
I Bulls and Cows får siffrorna inte upprepas, vilket de får göra i Mastermind.
I Mastermind används också (oftast) bara talen 1-6 medan talen 0-9 används 
i Bulls and Cows.


\section{Syfte}
\noindent
Syftet är att fördjupa dina kunskaper inom och utveckla din vana för 
programmering.
I detta projekt kommer du att fokusera på algoritmer, datastrukturer och i viss 
utsträckning representera dessa data i filer.


\section{Teori}
\label{sec:Theory}
\noindent
\citet{Knuth1977tca} föreslog något år efter att spelet Mastermind först 
släptes en algoritm för att knäcka koden med som mest fem försök.
Många andra har också föreslagit olika algoritmer för att lösa problemet att 
knäcka koden 
\cite[exempelvis][]{Neuwirth1982ssf,Vomlel2004bni,Goddard2004mr,Kooi2005yam,Slovesnov2011soo}.
I detta projekt ska du till en början titta på en enkel algoritm för att lösa 
problemet, den algoritm du ska använda dig av ges 
i \prettyref{alg:bullcow-simple}.

\begin{algorithm}[t]
	\begin{algorithmic}[1]
		\Require{%
			En mängd \( G = \{ (g_i, a_i) \} \) av gissningar \(g_i\) med respektive 
			svar \(a_i\).
		}
		\Ensure{%
			En ny gissning \(g\) som inte redan finns i \(G\).
		}
		\Statex
		\Function{BullsCows}{$x, y$}
			\State\Return Antalet \emph{bulls} och \emph{cows} när \(x\) jämförs med 
			\(y\).
			\Comment Detta ger samma resultat oavsett om \(x\) eller \(y\) används 
			som hemligt värde.
		\EndFunction
		\Statex
		\Function{GiltigGissning}{$g, G$}
			\ForAll{$g_i\in G$}
				\If{\Call{BullsCows}{$g, g_i$} $\neq a_i$}
					\State \Return falskt
				\EndIf
			\EndFor
			\State \Return sant
		\EndFunction
		\Statex
		\Function{GenereraGissning}{$G$}
			\If{$G = \emptyset$}\Comment Detta garanterar att \(g\not\in G\).
				\State Låt \(g = 1023\).
			\Else
				\State Låt \(g\) vara den största gissningen i \(G\) adderad med ett.
			\EndIf
			\While{\Call{GiltigGissning}{$g, G$} = falskt}\Comment Genomför detta 
			tills att vi hittar en giltig gissning.
				\State Öka \(g\) till nästa tal med distinkta siffror.
			\EndWhile
			\State \Return \(g\)
		\EndFunction
	\end{algorithmic}
  \caption{En enkel algoritm för att generera en gissning för spelet Bulls and 
  Cows.}
	\label{alg:bullcow-simple}
\end{algorithm}

Algoritm \ref{alg:bullcow-simple} har funktionen \textsc{GenereraGissning} för 
att generera en gissning utifrån tidigare gissningar, eller en initial gissning 
om inga tidigare gissningar finns.
Algoritmen garanteras avslut om gissningarna och svaren i mängden \(G\) stämmer 
överens.
Således, om kodmakaren fuskar genom att under spelets gång byta sitt hemliga 
tal är resultatet av \prettyref{alg:bullcow-simple} odefinierat.


\section{Uppgift}
\label{sec:Tasks}
\noindent
Projektet är uppdelat i olika betygssteg.
För betyget D krävs att du genomför grunduppgiften utan anmärkningar 
enligt granskningsprotokollet.
Vid maximalt två anmärkningar ges betyget E, vid fler anmärkningar betyget F.

För betygen C-A krävs de extrauppgifter som ges efter grunduppgiften, för ett 
betyg krävs att samtliga föregående extrauppgifter även är genomförda.
(Även dessa ska vara utan anmärkningar, vid anmärkning sänks betyget 
fortfarande till E.)


\subsection{Grunduppgift}
\noindent
I grundutförandet ska du implementera en version av spelet där programmet 
gissar på talet som användaren tänker på.
Det vill säga programmet agerar kodknäckare och användaren kodmakare.
Detta kan du göra genom att implementera \prettyref{alg:bullcow-simple}.
Användaren tänker alltså på ett tal, programmet skriver ut gissningarna på 
skärmen och användaren matar in antalet bulls och cows.

\subsection{Extrauppgift C}
\noindent
Ditt program ska ha felhantering.
Det ska kunna detektera när användaren har fuskat (eller bara tänkt fel) för 
att programmet inte ska fortsätta i all evighet.
En lösning i likhet med att införa en maxgräns på 1000 gissningar är dock inte 
giltig, programmet ska säga till precis när det går att avgöra att användaren 
gjort fel -- helst med en kommentar i stil med ''Ye cheatin' scum!''.

Med felhantering menas även att programmet ska hantera om en användare matar in 
fel, exempelvis om denne matar in bokstäver där det förväntas heltal.
Sammanfattningsvis, en användare ska inte kunna krascha programmet.

\subsection{Extrauppgift B}
\noindent
Utöka ditt program så att användaren kan agera kodknäckare och programmet är 
kodmakare, det vill säga ombytta roller jämfört med grunduppgiften.
Du ska även skapa en highscorelista.

\subsection{Extrauppgift A}
\noindent
I denna extrauppgift ges flera alternativ.
Du väljer ett av dessa som du implementerar i ditt program.
\begin{itemize}
	\item Använd funktionerna i ditt program för att låta programmet spela mot 
		sig självt och med hjälp av highscorefunktionen skapa statistik för vilka 
		tal som är svårast för algoritmen att knäcka.
	\item Implementera en förbättrad kodknäckningsalgoritm.
	\item Alla val ska kunna göras via argument till programmet från terminalen.
		Du kan använda biblioteket getopt(3).
		Exempelvis om den exekverbara filen heter \texttt{bullcow} och vi vill 
		specificera att highscorelistan ska lagras i filen \texttt{highscore.txt}:
		\begin{terminal}
$ ./bullcow -f highscore.txt
		\end{terminal}
\end{itemize}


%%% EXAMINATION %%%
\section{Examination}
\label{sec:Examination}
\noindent
Din lösning ska redovisas muntligen för en lärare vid något av kursens 
redovisningstillfällen.
När du redovisat och fått godkänt laddar du upp din källkod (och byggskript) 
till inlämningslådan i lärplattformen.



\bibliography{literature}
\end{document}
