% $Id$
% Author: Daniel Bosk <daniel.bosk@miun.se>
\documentclass[a4paper]{miunasgn}
\usepackage[utf8]{inputenc}
\usepackage[T1]{fontenc}
\usepackage[english,swedish]{babel}
\usepackage{url,hyperref}
\usepackage{prettyref,varioref}
\usepackage{natbib}
\usepackage{color,listings}
\usepackage[plain]{algorithm}
\usepackage{algpseudocode}
\usepackage{amsmath,amsthm,amssymb}
\usepackage[nofancy,today]{svninfo}
\usepackage[natbib,varioref,prettyref,listings,algorithm]{miunmisc}

\svnInfo $Id$
%\printanswers

\theoremstyle{definition}
\newtheorem{example}{Exempel}

\courseid{DT028G}
\course{Introduktion till programmering}
\assignmenttype{Projekt}
\title{Game of Life}
\author{Daniel Bosk\footnote{%
	Detta verk är tillgängliggjort under licensen Creative Commons 
	Erkännande-DelaLika 2.5 Sverige (CC BY-SA 2.5 SE).
	För att se en sammanfattning och kopia av licenstexten besök URL 
	\url{http://creativecommons.org/licenses/by-sa/2.5/se/}.
}}
\date{\svnId}

\begin{document}
\maketitle
\thispagestyle{foot}
\tableofcontents


\section{Introduktion}
\noindent
\dots


\section{Syfte}
\noindent
Syftet är att fördjupa dina kunskaper inom och utveckla din vana för 
programmering.


\section{Teori}
\label{sec:Theory}
\noindent
\dots


\section{Uppgift}
\label{sec:Tasks}
\noindent
Projektet är uppdelat i olika betygssteg.
För betyget D krävs att du genomför grunduppgiften utan anmärkningar 
enligt granskningsprotokollet.
Vid maximalt två anmärkningar ges betyget E, vid fler anmärkningar betyget F.

För betygen C-A krävs de extrauppgifter som ges efter grunduppgiften, för ett 
betyg krävs att samtliga föregående extrauppgifter även är genomförda.
(Även dessa ska vara utan anmärkningar, vid anmärkning sänks betyget 
fortfarande till E.)


\subsection{Grunduppgift}
\noindent
\dots

\subsection{Extrauppgift C}
\noindent
\dots

\subsection{Extrauppgift B}
\noindent
\dots

\subsection{Extrauppgift A}
\noindent
\dots


%%% EXAMINATION %%%
\section{Examination}
\label{sec:Examination}
\noindent
Din lösning ska redovisas muntligen för en lärare vid något av kursens 
redovisningstillfällen.
När du redovisat och fått godkänt laddar du upp din källkod (och byggskript) 
till inlämningslådan i lärplattformen.



\bibliography{literature}
\end{document}
