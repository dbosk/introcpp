% $Id$
% Author: Daniel Bosk <daniel.bosk@miun.se>
\documentclass[a4paper]{miunasgn}
\usepackage[utf8]{inputenc}
\usepackage[english,swedish]{babel}
\usepackage{url,hyperref}
\usepackage{prettyref,varioref}
\usepackage{natbib}
\usepackage{color,listings}
\usepackage[nofancy,today]{svninfo}
\usepackage[natbib,varioref,prettyref,listings]{miunmisc}

\svnInfo $Id$
%\printanswers

\courseid{DT028G}
\course{Introduktion till programmering}
\assignmenttype{Projekt}
\title{En kontaktbok}
\author{Daniel Bosk\footnote{%
	Detta verk är tillgängliggjort under licensen Creative Commons 
	Erkännande-DelaLika 2.5 Sverige (CC BY-SA 2.5 SE).
	För att se en sammanfattning och kopia av licenstexten besök URL 
	\url{http://creativecommons.org/licenses/by-sa/2.5/se/}.
}}
\date{\svnId}

\begin{document}
\maketitle
\thispagestyle{foot}
\tableofcontents


\section{Introduktion}
\noindent
Det är mycket information som ska hållas ordning på för den moderna människan.
Kontaktuppgifter för andra människor är en del av den.
Som hjälpmedel har vi exempelvis Contacts i Googles GMail, Facebook eller 
telefonböcker av alla de slag i mobiltelefonen.
I detta projekt ska du fördjupa dig i vad det krävs av ett program som ska 
hantera denna typ av information.


\section{Syfte}
\noindent
Syftet är att fördjupa dina kunskaper inom och utveckla din vana för 
programmering.
Du kommer att få fördjupade kunskaper om hur datastrukturerna som presenterats 
under kursens gång kan tillämpas.


\section{Teori}
\label{sec:Theory}
\noindent
De datastrukturer du kan ha nytta av i detta projekt är bland andra
\begin{itemize}
	\item \code{std::map} eller \code{std::multimap},
	\item \code{std::vector}, och
	\item \code{std::fstream}.
\end{itemize}


\section{Uppgift}
\label{sec:Tasks}
\noindent
Projektet är uppdelat i olika betygssteg.
För betyget D krävs att du genomför grunduppgiften utan anmärkningar 
enligt granskningsprotokollet.
Vid maximalt två anmärkningar ges betyget E, vid fler anmärkningar betyget F.

För betygen C-A krävs de extrauppgifter som ges efter grunduppgiften, för ett 
betyg krävs att samtliga föregående extrauppgifter även är genomförda.
(Även dessa ska vara utan anmärkningar, vid anmärkning sänks betyget 
fortfarande till E.)


\subsection{Grunduppgift}
\noindent
Skapa en grundläggande kontaktbok där man kan lagra och sedan ta fram 
information om kontakter.
Den information som (åtminstone) ska kunna lagras är
\begin{itemize}
	\item namn,
	\item (post-) adress,
	\item e-postadress,
	\item telefonnummer,
	\item födelsedag, och
	\item övrigt.
\end{itemize}

Gränssnittet ska fungera på följande sätt.
Användaren ska kunna välja vad denne vill göra, exempelvis lägga till en 
kontakt eller hitta telefonnumret för en given kontakt.
Det finns flera sätt att implementera detta, exempelvis:
\begin{itemize}
	\item Att ha en meny där användaren får välja genom att mata in ett tal 
		motsvarande ett av menyalternativen.
		Avsluta ska vara ett av alternativen.
  \item Att ha ett kommandodrivet gränssnitt där användaren ger kommandon för 
    vad som ska göras, exempelvis: ''add Adam 070-1234567''.
\end{itemize}

Den funktionalitet som ska finnas är
\begin{itemize}
	\item att lägga till en kontakt,
  \item att ta bort en kontakt, och
	\item att söka efter en kontakt.
\end{itemize}

Kontaktboken ska sparas i en fil för att den inte ska försvinna mellan 
körningarna av programmet.
(Det vore en tämligen meningslös kontaktbok om den inte kommer ihåg kontakterna 
längre än dess användare.)

\subsection{Extrauppgift C}
\noindent
Programmet ska inte kunna kraschas av användaren.
Exempelvis, om användaren matar in fel typ av data ska programmet be om ny data 
istället för att avslutas.
Eller om kontaktbokens fil är korrupt, då ska programmet tala om för användaren 
att något är fel och att kontaktboken inte kunde laddas.
Den ska alltså inte ge ''Gatvägen 20, 12345 Astad'' som telefonnummer, eller 
dylikt, för en kontakt.

\subsection{Extrauppgift A}
\noindent
Vid sökning ska programmet söka efter söksträngen i alla fält för kontakterna.
Matchningen ska också vara okänslig för om versaler eller gemener används.
Exempelvis om jag söker efter strängen \emph{ada}, då ska programmet lista 
kontakterna \emph{Ada Arvidsson}, \emph{Adam Allsing} och \emph{Cecilia 
Adamsdotter}.
Eftersom att det ska söka i alla fält ska det även lista alla kontakter som bor 
på \emph{Ada Lovelaces väg} eller som har \emph{Adamstorp} som postort.
Den ska även matcha delsträngar: en sökning på ''torp'' ska ge personer med 
postort ''Adamstorp'' i resultatet.



%%% EXAMINATION %%%
\section{Examination}
\label{sec:Examination}
\noindent
Din lösning ska redovisas muntligen för en lärare vid något av kursens 
redovisningstillfällen.
När du redovisat och fått godkänt laddar du upp din källkod (och byggskript) 
till inlämningslådan i lärplattformen.



%\bibliography{literature}
\end{document}
